% Options for packages loaded elsewhere
\PassOptionsToPackage{unicode}{hyperref}
\PassOptionsToPackage{hyphens}{url}
%
\documentclass[
  french,
]{article}
\usepackage{amsmath,amssymb}
\usepackage{lmodern}
\usepackage{ifxetex,ifluatex}
\ifnum 0\ifxetex 1\fi\ifluatex 1\fi=0 % if pdftex
  \usepackage[T1]{fontenc}
  \usepackage[utf8]{inputenc}
  \usepackage{textcomp} % provide euro and other symbols
\else % if luatex or xetex
  \usepackage{unicode-math}
  \defaultfontfeatures{Scale=MatchLowercase}
  \defaultfontfeatures[\rmfamily]{Ligatures=TeX,Scale=1}
\fi
% Use upquote if available, for straight quotes in verbatim environments
\IfFileExists{upquote.sty}{\usepackage{upquote}}{}
\IfFileExists{microtype.sty}{% use microtype if available
  \usepackage[]{microtype}
  \UseMicrotypeSet[protrusion]{basicmath} % disable protrusion for tt fonts
}{}
\makeatletter
\@ifundefined{KOMAClassName}{% if non-KOMA class
  \IfFileExists{parskip.sty}{%
    \usepackage{parskip}
  }{% else
    \setlength{\parindent}{0pt}
    \setlength{\parskip}{6pt plus 2pt minus 1pt}}
}{% if KOMA class
  \KOMAoptions{parskip=half}}
\makeatother
\usepackage{xcolor}
\IfFileExists{xurl.sty}{\usepackage{xurl}}{} % add URL line breaks if available
\IfFileExists{bookmark.sty}{\usepackage{bookmark}}{\usepackage{hyperref}}
\hypersetup{
  pdftitle={Entreprises autour de l'aéroport de Strasbourg},
  pdfauthor={Cantien Collinet},
  pdflang={fr},
  hidelinks,
  pdfcreator={LaTeX via pandoc}}
\urlstyle{same} % disable monospaced font for URLs
\usepackage[margin=1in]{geometry}
\usepackage{longtable,booktabs,array}
\usepackage{calc} % for calculating minipage widths
% Correct order of tables after \paragraph or \subparagraph
\usepackage{etoolbox}
\makeatletter
\patchcmd\longtable{\par}{\if@noskipsec\mbox{}\fi\par}{}{}
\makeatother
% Allow footnotes in longtable head/foot
\IfFileExists{footnotehyper.sty}{\usepackage{footnotehyper}}{\usepackage{footnote}}
\makesavenoteenv{longtable}
\usepackage{graphicx}
\makeatletter
\def\maxwidth{\ifdim\Gin@nat@width>\linewidth\linewidth\else\Gin@nat@width\fi}
\def\maxheight{\ifdim\Gin@nat@height>\textheight\textheight\else\Gin@nat@height\fi}
\makeatother
% Scale images if necessary, so that they will not overflow the page
% margins by default, and it is still possible to overwrite the defaults
% using explicit options in \includegraphics[width, height, ...]{}
\setkeys{Gin}{width=\maxwidth,height=\maxheight,keepaspectratio}
% Set default figure placement to htbp
\makeatletter
\def\fps@figure{htbp}
\makeatother
\setlength{\emergencystretch}{3em} % prevent overfull lines
\providecommand{\tightlist}{%
  \setlength{\itemsep}{0pt}\setlength{\parskip}{0pt}}
\setcounter{secnumdepth}{5}
\ifxetex
  % Load polyglossia as late as possible: uses bidi with RTL langages (e.g. Hebrew, Arabic)
  \usepackage{polyglossia}
  \setmainlanguage[]{french}
\else
  \usepackage[main=french]{babel}
% get rid of language-specific shorthands (see #6817):
\let\LanguageShortHands\languageshorthands
\def\languageshorthands#1{}
\fi
\ifluatex
  \usepackage{selnolig}  % disable illegal ligatures
\fi

\title{Entreprises autour de l'aéroport de Strasbourg}
\author{Cantien Collinet}
\date{4/6/2021}

\begin{document}
\maketitle

{
\setcounter{tocdepth}{2}
\tableofcontents
}
\hypertarget{constitution-de-la-base}{%
\section{Constitution de la base}\label{constitution-de-la-base}}

\hypertarget{fichier-sirene-des-uxe9tablissements-dans-la-zone}{%
\subsection{Fichier SIRENE des établissements dans la
zone}\label{fichier-sirene-des-uxe9tablissements-dans-la-zone}}

On commence par interroger la base SIRENE des établissements situés dans
les communes concernées :
\url{https://www.sirene.fr/sirene/public/creation-fichier}.

Les communes demandées sont :

\begin{itemize}
\tightlist
\item
  ENTZHEIM
\item
  HANGENBIETEN
\item
  DUTTLENHEIM
\item
  DUPPIGHEIM
\item
  ERNOLSHEIM-BRUCHE
\item
  HOLTZHEIM
\item
  WOLFISHEIM
\end{itemize}

Le fichier correspondant est placé dans
\texttt{data/tables/sirene-all.csv}. La liste des variables est
disponible sur :
\url{https://www.sirene.fr/sirene/public/static/liste-variables}.

\hypertarget{guxe9ocodage-du-fichier-sirene}{%
\subsection{Géocodage du fichier
SIRENE}\label{guxe9ocodage-du-fichier-sirene}}

On géocode le fichier grâce au service disponible sur
\url{https://adresse.data.gouv.fr/csv}. Critères :

\begin{itemize}
\tightlist
\item
  numeroVoieEtablissement
\item
  typeVoieEtablissement
\item
  libelleVoieEtablissement
\item
  complementAdresseEtablissement
\item
  codePostalEtablissement
\item
  libelleCommuneEtablissement
\end{itemize}

Paramètres avancés : code INSEE = codeCommuneEtablissement.

Le fichier correspondant est placé dans
\texttt{data/tables/sirene-all-geocoded.csv}.

\hypertarget{nettoyage-du-fichier-sirene}{%
\subsection{Nettoyage du fichier
SIRENE}\label{nettoyage-du-fichier-sirene}}

On importe le fichier \texttt{data/tables/sirene-all.geocoded.csv} dans
R, en ne garde que les colonnes qui nous intéressent.

Il y a 3929 établissements dans les communes de la zone.

\hypertarget{chiffres-daffaires-par-siren}{%
\subsection{Chiffres d'affaires par
SIREN}\label{chiffres-daffaires-par-siren}}

On récupère les chiffres clés 2020-2019-2018 des entreprises en France
sur
\url{https://opendata.datainfogreffe.fr/explore/dataset/chiffres-cles-2020}
et on les place dans \texttt{data/tables/chiffres-cles-2020.csv}. On
importe ce fichier dans R.

On fusionne les deux tables.

\hypertarget{filtres}{%
\subsection{Filtres}\label{filtres}}

On enlève les entreprises sans employés, les entreprises
unipersonnelles, les entreprises de droit public/administratif, les
établissements publics des cultes d'Alsace-Lorraine.

Il y a 892 établissements ciblés dans les communes de la zone.

\hypertarget{zones-dactivituxe9s-cibluxe9es}{%
\subsection{Zones d'activités
ciblées}\label{zones-dactivituxe9s-cibluxe9es}}

Dans qgis, on dessine les zones d'activité et on enregistre la couche
dans \texttt{data/spatial/zonesdactivite/zonesdactivite.shp}. Puis on
l'importe dans R et on demande les établissements qui sont dans les
polygones dessinés.

Il y a 529 établissements ciblés dans les zones d'activité étudiées.

\hypertarget{export}{%
\subsection{Export}\label{export}}

Le fichier des établissements de la zone est exporté dans le fichier
\texttt{data/tables/etablissements.xls}, celui des établissements des
zones d'activité dans \texttt{data/tables/etablissements-za.xls}.

\hypertarget{statistiques}{%
\section{Statistiques}\label{statistiques}}

\hypertarget{dans-les-communes-cibles}{%
\subsection{Dans les communes cibles}\label{dans-les-communes-cibles}}

\begin{longtable}[]{@{}
  >{\raggedright\arraybackslash}p{(\columnwidth - 2\tabcolsep) * \real{0.88}}
  >{\raggedright\arraybackslash}p{(\columnwidth - 2\tabcolsep) * \real{0.12}}@{}}
\toprule
\textbf{Characteristic} & \textbf{N = 892} \\
\midrule
\endhead
NAF niveau 1 & \\
Commerce ; réparation d'automobiles et de motocycles & 168 (19\%) \\
Construction & 130 (15\%) \\
Industrie manufacturière & 100 (11\%) \\
Activités spécialisées, scientifiques et techniques & 95 (11\%) \\
Activités de services administratifs et de soutien & 80 (9.0\%) \\
Transports et entreposage & 66 (7.4\%) \\
Information et communication & 52 (5.8\%) \\
Activités financières et d'assurance & 41 (4.6\%) \\
Hébergement et restauration & 40 (4.5\%) \\
Autres activités de services & 34 (3.8\%) \\
Santé humaine et action sociale & 25 (2.8\%) \\
Activités immobilières & 20 (2.2\%) \\
Enseignement & 12 (1.3\%) \\
Agriculture, sylviculture et pêche & 11 (1.2\%) \\
Arts, spectacles et activités récréatives & 9 (1.0\%) \\
Production et distribution d'électricité, de gaz, de vapeur et d'air
conditionné & 5 (0.6\%) \\
Industries extractives & 3 (0.3\%) \\
Production et distribution d'eau ; assainissement, gestion des déchets
et dépollution & 1 (0.1\%) \\
\bottomrule
\end{longtable}

\begin{longtable}[]{@{}
  >{\raggedright\arraybackslash}p{(\columnwidth - 2\tabcolsep) * \real{0.90}}
  >{\raggedright\arraybackslash}p{(\columnwidth - 2\tabcolsep) * \real{0.10}}@{}}
\toprule
\textbf{Characteristic} & \textbf{N = 892} \\
\midrule
\endhead
\textbf{Effectif établissement} & \\
0 salarié (n'ayant pas d'effectif au 31/12 mais ayant employé des
salariés au cours de l'année de référence) & 29 (4.2\%) \\
1 ou 2 salariés & 196 (28\%) \\
3 à 5 salariés & 126 (18\%) \\
6 à 9 salariés & 96 (14\%) \\
10 à 19 salariés & 102 (15\%) \\
20 à 49 salariés & 79 (11\%) \\
50 à 99 salariés & 17 (2.4\%) \\
100 à 199 salariés & 10 (1.4\%) \\
200 à 249 salariés & 0 (0\%) \\
250 à 499 salariés & 5 (0.7\%) \\
500 à 999 salariés & 1 (0.1\%) \\
1 000 à 1 999 salariés & 0 (0\%) \\
2 000 à 4 999 salariés & 0 (0\%) \\
5 000 à 9 999 salariés & 0 (0\%) \\
10 000 salariés et plus & 0 (0\%) \\
Etablissement non employeur (pas de salarié au cours de l'année de
référence et pas d'effectif au 31/12) & 34 (4.9\%) \\
Unknown & 197 \\
\textbf{Effectif unité légale} & \\
0 salarié (n'ayant pas d'effectif au 31/12 mais ayant employé des
salariés au cours de l'année de référence) & 30 (3.4\%) \\
1 ou 2 salariés & 211 (24\%) \\
3 à 5 salariés & 138 (15\%) \\
6 à 9 salariés & 95 (11\%) \\
10 à 19 salariés & 100 (11\%) \\
20 à 49 salariés & 107 (12\%) \\
50 à 99 salariés & 41 (4.6\%) \\
100 à 199 salariés & 43 (4.8\%) \\
200 à 249 salariés & 13 (1.5\%) \\
250 à 499 salariés & 26 (2.9\%) \\
500 à 999 salariés & 37 (4.1\%) \\
1 000 à 1 999 salariés & 21 (2.4\%) \\
2 000 à 4 999 salariés & 17 (1.9\%) \\
5 000 à 9 999 salariés & 4 (0.4\%) \\
10 000 salariés et plus & 9 (1.0\%) \\
Etablissement non employeur (pas de salarié au cours de l'année de
référence et pas d'effectif au 31/12) & 0 (0\%) \\
\bottomrule
\end{longtable}

CA 2019 des entreprises auxquelles appartiennent les établissements dans
les communés étudiées :

\begin{longtable}[]{@{}ll@{}}
\toprule
\textbf{Characteristic} & \textbf{N = 892} \\
\midrule
\endhead
\textbf{tranche\_ca\_millesime\_2} & \\
E + d 1M & 46 (57\%) \\
A - de 32K & 29 (36\%) \\
C entre 82K et 250K & 3 (3.7\%) \\
D entre 250K et 1M & 3 (3.7\%) \\
Unknown & 811 \\
\bottomrule
\end{longtable}

\hypertarget{dans-les-zones-dactivituxe9-cibles}{%
\subsection{Dans les zones d'activité
cibles}\label{dans-les-zones-dactivituxe9-cibles}}

\begin{longtable}[]{@{}ll@{}}
\toprule
\textbf{Characteristic} & \textbf{N = 529} \\
\midrule
\endhead
nom\_zone\_d\_activite & \\
aeroparc & 262 (50\%) \\
plaine bruche & 151 (29\%) \\
hantgebieten/entzheim & 61 (12\%) \\
joffre holtzheim & 55 (10\%) \\
\bottomrule
\end{longtable}

\begin{longtable}[]{@{}
  >{\raggedright\arraybackslash}p{(\columnwidth - 2\tabcolsep) * \real{0.88}}
  >{\raggedright\arraybackslash}p{(\columnwidth - 2\tabcolsep) * \real{0.12}}@{}}
\toprule
\textbf{Characteristic} & \textbf{N = 529} \\
\midrule
\endhead
NAF niveau 1 & \\
Commerce ; réparation d'automobiles et de motocycles & 115 (22\%) \\
Industrie manufacturière & 76 (14\%) \\
Construction & 72 (14\%) \\
Activités spécialisées, scientifiques et techniques & 65 (12\%) \\
Activités de services administratifs et de soutien & 42 (7.9\%) \\
Information et communication & 42 (7.9\%) \\
Transports et entreposage & 31 (5.9\%) \\
Activités financières et d'assurance & 28 (5.3\%) \\
Activités immobilières & 13 (2.5\%) \\
Hébergement et restauration & 13 (2.5\%) \\
Autres activités de services & 12 (2.3\%) \\
Santé humaine et action sociale & 6 (1.1\%) \\
Enseignement & 4 (0.8\%) \\
Arts, spectacles et activités récréatives & 3 (0.6\%) \\
Industries extractives & 3 (0.6\%) \\
Production et distribution d'électricité, de gaz, de vapeur et d'air
conditionné & 2 (0.4\%) \\
Agriculture, sylviculture et pêche & 1 (0.2\%) \\
Production et distribution d'eau ; assainissement, gestion des déchets
et dépollution & 1 (0.2\%) \\
\bottomrule
\end{longtable}

\begin{longtable}[]{@{}
  >{\raggedright\arraybackslash}p{(\columnwidth - 2\tabcolsep) * \real{0.90}}
  >{\raggedright\arraybackslash}p{(\columnwidth - 2\tabcolsep) * \real{0.10}}@{}}
\toprule
\textbf{Characteristic} & \textbf{N = 529} \\
\midrule
\endhead
\textbf{Effectif établissement} & \\
0 salarié (n'ayant pas d'effectif au 31/12 mais ayant employé des
salariés au cours de l'année de référence) & 16 (4.0\%) \\
1 ou 2 salariés & 75 (19\%) \\
3 à 5 salariés & 68 (17\%) \\
6 à 9 salariés & 62 (16\%) \\
10 à 19 salariés & 80 (20\%) \\
20 à 49 salariés & 59 (15\%) \\
50 à 99 salariés & 15 (3.8\%) \\
100 à 199 salariés & 7 (1.8\%) \\
200 à 249 salariés & 0 (0\%) \\
250 à 499 salariés & 4 (1.0\%) \\
500 à 999 salariés & 1 (0.3\%) \\
1 000 à 1 999 salariés & 0 (0\%) \\
2 000 à 4 999 salariés & 0 (0\%) \\
5 000 à 9 999 salariés & 0 (0\%) \\
10 000 salariés et plus & 0 (0\%) \\
Etablissement non employeur (pas de salarié au cours de l'année de
référence et pas d'effectif au 31/12) & 12 (3.0\%) \\
Unknown & 130 \\
\textbf{Effectif unité légale} & \\
0 salarié (n'ayant pas d'effectif au 31/12 mais ayant employé des
salariés au cours de l'année de référence) & 19 (3.6\%) \\
1 ou 2 salariés & 78 (15\%) \\
3 à 5 salariés & 72 (14\%) \\
6 à 9 salariés & 56 (11\%) \\
10 à 19 salariés & 76 (14\%) \\
20 à 49 salariés & 75 (14\%) \\
50 à 99 salariés & 35 (6.6\%) \\
100 à 199 salariés & 31 (5.9\%) \\
200 à 249 salariés & 9 (1.7\%) \\
250 à 499 salariés & 19 (3.6\%) \\
500 à 999 salariés & 26 (4.9\%) \\
1 000 à 1 999 salariés & 14 (2.6\%) \\
2 000 à 4 999 salariés & 12 (2.3\%) \\
5 000 à 9 999 salariés & 3 (0.6\%) \\
10 000 salariés et plus & 4 (0.8\%) \\
Etablissement non employeur (pas de salarié au cours de l'année de
référence et pas d'effectif au 31/12) & 0 (0\%) \\
\bottomrule
\end{longtable}

CA 2019 des entreprises auxquelles appartiennent les établissements dans
les communés étudiées :

\begin{longtable}[]{@{}ll@{}}
\toprule
\textbf{Characteristic} & \textbf{N = 529} \\
\midrule
\endhead
\textbf{tranche\_ca\_millesime\_2} & \\
E + d 1M & 29 (62\%) \\
A - de 32K & 15 (32\%) \\
C entre 82K et 250K & 2 (4.3\%) \\
D entre 250K et 1M & 1 (2.1\%) \\
Unknown & 482 \\
\bottomrule
\end{longtable}

\end{document}
